\title{\fontsize{11}{11}\selectfont Министерство науки и высшего образования
Республики Казахстан \\
\hfill \break
Карагандинский университет имени академика Е.А.Букетова \\
\hfill \break
Факультет \faculty \\
\hfill \break
Кафедра \department \\
\vspace{40mm}
\textbf{РАБОЧАЯ УЧЕБНАЯ ПРОГРАММА - SYLLABUS} \\
\hfill \break
по дисциплине \\
\hfill \break
"\subject" \\

\hfill \break
для образовательной программы \\
\hfill \break
"\group"
}
\maketitle
\vspace{30mm}
\raggedright
Курс: \course \\
Срок обучения: \studyTime \\
Объем кредитов: \credits \\

\vfill

Одобрено на заседании кафедры, \\
протокол \\
\hfill \break
Утверждено на заседании  Комиссии факультета по обеспечению качества, \\
протокол \\

\newpage
\section{\selectfont ТЕМАТИЧЕСКИЙ ПЛАН КУРСА}
\tematicsTable

\section{ДАННЫЕ О ПРЕПОДАВАТЕЛЕ (-ЯХ)}
\justifying
\lecturers

\section{ПОЛИТИКА КУРСА}
\justifying

Данный учебный курс организован в соответствии с требованиями Академической политики Карагандинского университета им. Е.А.Букетова. Настоятельно рекомендуется обратить внимание на возможные последствия в случае невыполнения академических требований и низкой посещаемости занятий. В ходе изучения данной дисциплины преподаватель и обучающиеся должны следовать принятым в университете Правилам академической честности. Академическая политика университета и Правила академической честности  находятся в свободном доступе на сайте Карагандинского университета им. Е.А.Букетова www.buketov.edu.kz, а также в Личном кабинете обучающихся. Всем студентам/магистрантам/докторантам предоставляются равные возможности участия в обсуждении учебных тем на занятиях. Все имеют права задавать вопросы и получать ответы по заявленным в Силлабусе учебным темам. Приветствуется оригинальность мышления, творческий подход обучающихся при выполнении заданий преподавалеля. От всех обучающихся требуется соблюдение академической культуры поведения, демонстрации взаимного уважения друг к другу. Обучающиеся с особыми образовательными потребностями могут воспользоваться правом на индивидуальный подход в обучении.

\section{ПРЕРЕКВИЗИТЫ КУРСА}
\justifying
\prerequisites

\section{ПОСТРЕКВИЗИТЫ КУРСА}
\justifying
\postRequisites

\section{КРАТКОЕ ОПИСАНИЕ КУРСА}
\justifying
\courseDescription

\section{РЕЗУЛЬТАТЫ ОБУЧЕНИЯ И МЕТОДЫ ОЦЕНКИ ИХ ДОСТИЖИМОСТИ}
\losAndMethodsTable

\section{МЕТОДЫ ОБУЧЕНИЯ}
\justifying
\teachingMethods

\section{МЕТОДЫ ОЦЕНКИ РЕЗУЛЬТАТОВ ОБУЧЕНИЯ}
\justifying
\gradeMethods

\section{ПЕРЕЧЕНЬ РЕКОМЕНДУЕМЫХ ИСТОЧНИКОВ ПО КУРСУ}
\justifying
\blAlText

\section{ПЛАН ЛЕКЦИОННЫХ ЗАНЯТИЙ}
\justifying
\lssLpsText

\section{ПЛАН СЕМИНАРСКИХ/ ПРАКТИЧЕСКИХ ЗАНЯТИЙ }
\justifying
\splpText

\section{ПЛАН ЛАБОРАТОРНЫХ ЗАНЯТИЙ}
\justifying
\labText

\section{ЗАДАНИЯ ДЛЯ САМОСТОЯТЕЛЬНОЙ РАБОТЫ ОБУЧАЮЩЕГОСЯ ПОД РУКОВОДСТВОМ ПРЕПОДАВАТЕЛЯ – СРОП}
\justifying
\sropText

\section{ЗАДАНИЯ ДЛЯ САМОСТОЯТЕЛЬНОЙ РАБОТЫ ОБУЧАЮЩЕГОСЯ – СРО}
\justifying
\sroText

\subsection{ТЕМАТИКА ПИСЬМЕННЫХ РАБОТ}
\justifying
\pwsText

\section{ПОЛИТИКА ОЦЕНИВАНИЯ}
\justifying
Обучающийся оценивается по результатам выполнения текущих и рубежных контрольных заданий, а также экзамена. По результатам оценивания формируется рейтинг обучающе-гося по 100-балльной шкале. Для получения допуска к экзаме-ну, обучающему необходимо набрать не менее 50 баллов. Шкала оценки размещена в Приложении 8 к «Академической политике Карагандинского университета им. Е.А.Букетова»
\gpText

\section{КРИТЕРИИ СУММАРНОГО ОЦЕНИВАНИЯ ОБУЧАЮЩИХСЯ ЗА ВЫПОЛНЕНИЕ ЗАДАНИЙ}
\justifying
\begin{center}
\begin{tabularx}{ \textwidth }
{
| >{\centering\arraybackslash}X
| >{\centering\arraybackslash}X
| >{\centering\arraybackslash}X
| >{\centering\arraybackslash}X |
}
 \hline
Буквенное значение & Цифровой эквивалент & Процентный эквиволент & Критерии оценивания в контексте дисциплины\\
 \hline
 A & 4,0 & 95-100 & \acAText \\
 \hline
 A- & 3,67 & 90-94 & \acAMinusText \\
 \hline
 B+ & 3,33 & 85-89 & \acBPlusText \\
 \hline
 B & 3,0 & 80-84 & \acBText \\
 \hline
 B- & 2,67 & 75-79 & \acBMinusText \\
 \hline
 C+ & 2,33 & 70-74 & \acCPlusText \\
 \hline
 C & 2,0 & 65-69 & \acCText \\
 \hline
 C- & 1,67 & 60-64 & \acCMinusText \\
 \hline
 D+ & 1,33 & 55-59 & \acDPlusText \\
 \hline
 D & 1,0 & 50-54 & \acDText \\
 \hline
 Fx & 0,5 & 25-49 & \acFxText \\
 \hline
 F & 0 & 0-24 & \acFText \\
 \hline

\end{tabularx}
\end{center}

\section*{СОДЕРЖАНИЕ СИЛЛАБУСА}
\justifying
\begin{center}
\begin{tabularx}{ \textwidth }
{
| >{\centering\arraybackslash}c
| >{\centering\arraybackslash}X
| >{\centering\arraybackslash}X |
}
 \hline
№ & Название раздела & Страница\\
 \hline
 1 & 4 & 5 \\
 \hline

\end{tabularx}
\end{center}